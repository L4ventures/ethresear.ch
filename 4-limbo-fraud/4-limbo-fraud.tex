\documentclass{article}
\usepackage[margin=0.7in]{geometry}
\usepackage[parfill]{parskip}
\usepackage[utf8]{inputenc}
\usepackage{amsmath,amssymb,amsfonts,amsthm}
\usepackage{csquotes}
\MakeOuterQuote{"}

\begin{document}

The original limbo exit described at https://ethresear.ch/t/reliable-exits-of-withheld-in-flight-transactions-limbo-exits/ solves the problem that a malicious operator could force a user in certain circumstances to go through 2 exit games to withdraw his coin. However, an analogous problem appears in this situation

<img src='/uploads/default/original/2X/8/84691fe0f2f322878e8775b1f8a92d74acc4caaf.jpg'>

in which a transaction spending C comitting to inclusion in B is sent, but B is withheld. We would like to cancel the fraudulent exit. Currently C must be submitted to try to challenge the exit, and if the challenge is cancelled, the cancellation reveals that C was spent in B. In that case, the revealed child of C can be submitted to actually cancel the fraudulent exit.

We can use the same technique as limbo exits to avoid the need for two challenges. We start by viewing a limbo exit as an explicit claim that the rightful owner (ie the result of the CFCR defined by the non-limbo-exit-CFCR) is contained in a set of two directly lineal coins. We can use a similar claim in the exit game.

Here is the full game spelled out, modifying the exit game from https://ethresear.ch/t/plasma-cash-with-smaller-exit-procedure-and-a-general-approach-to-safety-proofs ; there are two possible exits. For clarity "providing a proof of a transaction" means providing a merkle proof from the transaction root of a comitted plasma block, while "providing a transaction" just means providing a digitally signed message.

**Type 1 Exit**

1. Anyone can exit their coin by providing a proof of a transaction in their coin's history, say $C$
2. An exit can be challenged in three ways: i) by providing a proof of a transaction spending $C$, or ii) by providing a transaction $C^*$ in the coin's history before $C$, or iii) providing a proof of a transaction $C^*$ in the coin's history before $C$ and a transaction $T$ spending $C^*$ included in a block before $C$
3. A type (i) challenge cancels the exit immediately. a type (ii) challenge can be cancelled by providing the direct child of $C^*$. a type (iii) challenge can be cancelled by providing the direct child of $C^*$ which is not the output of $T$, or by providing a committed transaction spending the output of $T$.

**Type 2 Exit**

1. Anyone can exit their coin by providing a proof of a transaction in their coin's history, say $C$, and a transaction $T'$ spending $C$
2. An exit can be challenged in four ways: i) by providing a proof of a transaction different from $T'$ spending $C$, or ii) by providing a transaction $C^*$ in the coin's history before C, or iii) providing a proof of a transaction $C^*$ in the coin's history before $C$ and a transaction $T$ spending $C^*$ committing to a block before $C$, or iv) providing a committed transaction spending the output of $T'$.
3. Same as (3) in type 1 exits; additionally, type (iv) challenge cancels the exit immediately.

\end{document}
